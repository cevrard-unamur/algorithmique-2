\documentclass{article}
\usepackage[utf8]{inputenc}
\usepackage[margin=1in]{geometry}

\usepackage[T1]{fontenc}
\usepackage{inconsolata}

\usepackage{color}

\definecolor{pblue}{rgb}{0.13,0.13,1}
\definecolor{pgreen}{rgb}{0,0.5,0}
\definecolor{pred}{rgb}{0.9,0,0}
\definecolor{pgrey}{rgb}{0.46,0.45,0.48}

\usepackage{listings}
\lstset{language=Java,
  showspaces=false,
  showtabs=false,
  breaklines=true,
  showstringspaces=false,
  breakatwhitespace=true,
  commentstyle=\color{pgreen},
  keywordstyle=\color{pblue},
  stringstyle=\color{pred},
  basicstyle=\ttfamily,
  moredelim=[il][\textcolor{pgrey}]{$$},
  moredelim=[is][\textcolor{pgrey}]{\%\%}{\%\%},
  tabsize=4,
  literate={\ \ }{{\ }}1
}

\title{IHDCB331 - Algorithmique II\\
	\large{Devoir 3}
}
\author{Cédric Evrard}
\begin{document}

\maketitle

\section{Question 1}
\begin{enumerate}
	\item Tous les éléments de a[i], a[i + 1], ... a[j] sont égaux à zéro
\begin{lstlisting}
/*@ invariant (\forall int i;
  @             i >= 0 && i < j;
  @             a[i] == 0);
  @*/
\end{lstlisting}
	\item Tous les éléments de a sont distincts
\begin{lstlisting}
 /*@ invariant (\forall int i;
   @             i >= 0 && i < arrayDistinct.length;
   @             (\forall int j; j >= 0 && j < a && i != j; a[j] != a[i]));
   @*/
\end{lstlisting}
	\item Tous les nombres de a sont pairs
\begin{lstlisting}
 /*@ invariant (\forall int i;
   @             i >= 0 && i < a;
   @             a[i] % 2 == 0);
   @*/
\end{lstlisting}
	\item Tous les nombres de a sont inférieurs ou égaux à 2
\begin{lstlisting}
 /*@ invariant (\forall int i;
   @             i >= 0 && i < a.length;
   @             a[i] <= 2);
   @*/
\end{lstlisting}
	\item Tous les nombres pairs de a[i], a[i + 1], ... a[j] sont inférieurs à 10
\begin{lstlisting}
/*@ invariant (\forall int i;
   @             i >= 0 && i < j;
   @             (a[i] % 2 == 0 && a[i] < 10) || a[i] % 2 != 0);
   @*/
\end{lstlisting}
	\item Il existe une valeur zéro dans a[i], a[i + 1], ... a[j]
\begin{lstlisting}
 /*@ invariant (\exists int i;
   @             i >= 0 && i < j;
   @             a[i] == 0);
   @*/
\end{lstlisting}
	\item Les éléments de a sont triés par ordre croissant
\begin{lstlisting}
 /*@ invariant (\forall int i;
   @             i >= 0 && i < a - 1;
   @             a[i] <= a[i + 1]);
   @*/
\end{lstlisting}
	\item x est le minium de a
\begin{lstlisting}
/*@ normal_behavior
  @ assignale \nothing;
  @ ensures \forall (int i; i >= 0 && i < a.length; a[i] >= \result);
  @ ensures \exists (int i; i >= 0 && i < a.length; a[i] == \result);
  @*/
\end{lstlisting}
\end{enumerate}

\section{Question 2}
\begin{enumerate}
	\item \textbf{double} max(\textbf{double} x, \textbf{double} y)
\begin{lstlisting}
/*@ normal_behavior
  @ assignable \nothing;
  @ ensures (a > b && \result == a) || (a <= b && \result == b);
  @*/
\end{lstlisting}
	\item \textbf{boolean} contient(\textbf{double} [] a, \textbf{double} x)
\begin{lstlisting}
/*@ normal_behavior
  @ assignable \nothing;
  @ ensures \result == (\exists int i;
  @		i >= 0 && i < a.length;
  @		x == a[i]);
  @*/
\end{lstlisting}
	\item \textbf{double} max(\textbf{double} [] a)
\begin{lstlisting}
/*@ normal_behavior
  @ assignale \nothing;
  @ ensures \forall (int i; i >= 0 && i < a.length; a[i] <= \result);
  @ ensures \exists (int i; i >= 0 && i < a.length; a[i] == \result);
  @*/
\end{lstlisting}
	\item \textbf{int} factorielle(\textbf{int} n)
\begin{lstlisting}
/*@ normal_behavior
  @ assignable \nothing;
  @ requires n >= 1;
  @ ensures \result == (\product int i; i >= 1 && i <= n; i);
  @*/
\end{lstlisting}
	\item \textbf{int} intSqrt(\textbf{int} n)
\begin{lstlisting}
/*@ normal_behavior
  @ assignable \nothing;
  @ requires n >= 0;
  @ ensures \result * \result <= n;
  @ ensures (\result + 1) * (\result + 1) > n;
  @ ensures \result >= 0
  @*/
\end{lstlisting}
\end{enumerate}

\section{Question 10}
La question 10 se base sur la signature de la méthode suivante : \textit{public boolean enLigne(int[] a, int n, int x)} où le tableau de \textbf{String} à été remplacé par un tableau de \textbf{int}
\begin{lstlisting}
/*@ normal_behavior
  @ assignable \nothing;
  @ ensures \result == (\exists int i; i >= 0 && i < a.length - n;
  @		(\forall int j; j >= i && j <= i + n; a[j] == x)) && 
  @	 	(i == 0 || (i > 0 && a[i - 1] != x)) &&
  @	 	(i + n + 1 == a.length || (i + n + 1 < a.length && a[i + n + 1] != x);
  @*/
\end{lstlisting}

\section{Question 12}
\begin{lstlisting}
/*@ normal_behavior
  @ assignable \nothing;
  @ ensures \result == (\sum int i; i >= 0 && i < l; a[i]);
  @*/
public boolean int somme(int[] a, int l) {
	int somme = 0;
	int i = 0;
	
	/*@ loop_invariant i >= 0 && i < l;
	  @ loop_invariant somme == (\sum int j; i >= 0 && j < i; a[j]);
	  @ decreases l - i;
	  @*/
	while (i < l) {
		somme = somme + a[i];
		i = i + 1;
	}
	
	return somme;
}
\end{lstlisting}

\section{Question 13}
\begin{lstlisting}
/*@ normal_behavior
  @ assignable \nothing;
  @ ensures \result == m + n
  @*/
public int somme (int m, int n) {
	int res = m;
	int i = 0;
	
	/*@ loop_invariant i >= 0 && i < n;
	  @ loop_invariant res == \old(res) + 1 && i = \old(i) + 1;
	  @ decreases n - i
	  @*/
	while (i < n) {
		res = res + 1;
		i = i + 1;
	}
	
	return res; // Correction par rapport a l'enonce ou une variable result qui n'existe pas est retournee
}	
\end{lstlisting}

\section{Question 15}
\begin{lstlisting}
/*@ normal_behavior
  @ assignable nothing;
  @ requires true;
  @ ensures (a > b && \result == a) || (a <= b && \result == b)
  @*/
int max(int a, int b) {
	if (a > b) {
		return a;
	} else {
		return b;
	}
}
\end{lstlisting}

\section{Question 22}
\begin{lstlisting}
public class Tri {
	
	/*@ normal_behavior
	  @ assignable a;
	  @ ensures \forall int i; i >= 0 && i < a.length - 1; a[i] < a[i + 1]
	  @*/
	static void tri(int[] a) {
		boolean tri;
		
		/*@ loop_invariant trie == true && \forall int j; j >= 0 && j < a.length - 1; a[j] < a[j + 1]
		  @ loop_invariant trie == false && \exists int j; j >= 0 && j < a.length - 1; a[j] > a[j + 1]
		  @*/
		do {
			trie = true;
			
			/*@ loop_invariant i >= 0 && i <= a.length - 1;
		  	  @ loop_invariant trie == true && \forall int j; j >= 0 && j < i - 1; a[j] < a[j + 1]
		  	  @ loop_invariant trie == false && \exists int j; j >= 0 && j < i - 1; a[j] > a[j + 1]
		  	  @*/
			for (int i = 0; i < a.length - 1; i++) {
				if (a[i] > a[i + 1]) {
					/* ... */
				}
			}
		} while (!trie)
	}
} 
\end{lstlisting}
\end{document}
